\documentclass[9pt]{article}

\usepackage[a4paper,left=35mm,top=26mm,right=26mm,bottom=15mm]{geometry}

\usepackage{fullpage}
\usepackage{tabularx}
\usepackage{caption}
\usepackage{microtype}

\setlength{\parindent}{0em}
\setlength{\parskip}{0.6em}


\title{ARM Project Report}
\author{Group 9\\Ben Sheng Tan, Adanna Akwataghibe, Lan Yi and Ching Yin Wong}
\date{June 10 2016}

\begin{document}

\maketitle

\section{Implementation and Structure of Assembler}

The assembler was done in a similar way to the emulator, where everyone would work on separate functions and consolidate on more challenging portions. We use two-pass assembly process as outlined in the specification. We create a mapping from labels to addresses and we find the size of the program we are about to assemble. Later on, we tokenise each instruction, process it and store it in a big array representing memory. If the instruction requires to store additional big constant, we save it after the end of program. In assemble.c, we write the array into a file using our binary writer.\\

Both the emulator and assembler passed all the test cases in the given test suite. In library, the header files (arm11.h, instruction.h and bitwise.h) are shared between emulate.c and assemble.c. To support all the features needed in assemble.c, some additional functions are added in instruction.h.

\medskip

\begin{tabular}{ |p{5cm}|p{10cm}| } 
\hline
\multicolumn{2}{|c|}{src/assemble.c} \\
\hline
Macro & contains global variables used in the file with inclusive of header files implemented in own created library. \\ 
\hline
Function Prototype &  declare all the functions written in the file and initialise the function arrays used to call the proper function in running the instructions. \\ 
\hline
Binary File Reader &  contains the function to read and write the binary file. \\ 
\hline
Core & contains the functions in initialising the function arrays for calling the proper instructions' functions \\
\hline
Main & contains main function for running the file \\
\hline
Shifting & functions for initialising the constant integer, shifted registers and checking if operand is an expression or a register and shift register. \\
\hline
Instructions & contains all the necessary functions for executing each type of instruction \\
\hline
\end{tabular}

\medskip

\begin{tabular}{ |p{5cm}|p{10cm}|  }
\hline
\multicolumn{2}{|c|}{ src/library  } \\
\hline
instruction.h & (additional) contains the functions prototype and functions for the conversion between mnemonic to string and between string to numerical value \\
\hline
assembler.h & defines the structure of binary and assembler program and declares  functions written in implementation file\\
 \hline
assembler.c & functions for checking errors, freeing program, writing and generating a bin file and running the assemble program\\
\hline
tokens.h & defines the structure of token and declares  functions written in implementation file\\
\hline
tokens.c &  essential functions for running tokenize \ \\
\hline
symbolTableList.h & defines the structure of symbol table and declares  functions written in implementation file \\
\hline
symbolTableList.c &  functions for ADT Double-linked lists \\
\hline
\end{tabular}
\pagebreak
\section{Extension} \\
    \centerline{\hl{\textbf{Assembler Comments}}}  \\\\\
    We expanded the uses  the assembler so that the programmer would be able to read comments if they were added to the instructions. For example, an instruction could be ldr r1,
    \hl{\textbf{\#4,\emph{(careful using this instruction)}}} . The comments are a means to communicate directly to the programmer. In order to implement this new addition, we made an Abstract Data Type which would store the comments. We decided on a queue because it had a first in-first out execution. We decided to make the maximum capacity of the queue (or the number of comments) to be 50. This was because, we assume that wouldn’t than 50 instructions. \\\\\\
    \centerline{\hl{\textbf{Calender Game}}}  \\
    \begin{description}
    \item[Aim] \hfill 
        \begin{itemize}
            \item For the players to learn more about historical dates whilst having fun.
            \item For the players general enjoyment
        \end{itemize}
    \item[The Game] \hfill \\\\
    There are three modes for the players to play in :
        \begin{itemize}
            \item Date game 
            \item History game
            \item Hangman
        \end{itemize}
    \end{description}     
    The Date game consists of generating random dates and the player has to guess the day that each date falls on. We also added some additional questions which asks about the day of holidays.\\
    The History game consists on having questions about historical events and the player guessing the dates of these. It has several level of difficulty i.e. easy, medium and hard level, each having four, four and two questions respectively. And when the player gets at least two question in a level correct, then they are allowed to move on to the next level. The players also get points as they answer each question correctly – at the end of the questions, they could be able to get a maximum of 10 points. \\
    
    Hangman consists of the player answering questions related to historical events. In each question, the player has a total of 5 lives. The player has to get the answer to the question by guessing a letter each time. If they guess the right letter, they can guess again and so on. However, if the player guesses the wrong letter, they add a body part to the drawing of the hangman which implicitly decreases the number of lives of the player. Similarly to the other games, there are ten questions available to answer.\\
    
    \begin{description}
    \item[How to play:] \hfill \\
        \begin{itemize}
            \item Date game \\\\
            The player should guess (or calculate) the day of the week of the given date.\\
            \item History game \\\\
            The player should answer the questions with a date of the historical event.\\
            \item Hangman \\\\
            In each question, the player should guess the letter to get to the answer.\\
        \end{itemize}
   
 \item[How it was implemented:] \hfill \\
        \begin{itemize}
            \item Date game \\\\
            Generate dates for questions using random generator and calculate the answer using mathematical method. Calculated answer is then compared with the players input using scan and if the same, then congratulate player. Otherwise, the player is alerted that this is the wrong answer and the correct result is shown. After each question, we ask whether the player wants to keep playing or quit the game.\\
            \item History game \\\\
            We prepare ten questions for the player. We store the answers and compare the input of the player similar to Date game. The player starts in the easy level, and each time, their answer matched our stored answer then a point is added to the global variable which keeps score of the number of correct questions. We use the number of points to later determine whether the player can go to the next level. At the end of each question and level, we also ask the player if they would like to keep playing or quit.\\
            \item Hangman \\\\
            Similar to history game, we prepared questions and stored the answers for the players. This time, we gave the players dashes, to represent the answer and also the number of letters that the answer consists of. When each question is asked, the player is prompted to write a letter.  If the letter that the player inputs is contained in the answer, then the dashes are updated (using checkanswer method). However, if the letter is not in the answer, then we go to the method (showhangman) and we select the case according to the number of fails (global variable) of the player so far in the question. If the question is gotten without the tries of the player being exhausted (showhangman is in case 1 where hangman is fully drawn), then the player is alerted that they have gotten the right answer. If not, the player is told that the hangman is hanged. \\
        \end{itemize}   
    \end{description}
    In all the games we have also done a lot of error handling to try and fix cases where the user has inputed the wrong kind of input. \\

 
  \begin{table}[h]
        \centering
      \caption*{\bf Additional instructions}
    \end{table}
   
    We implemented the Block Data Transfer Instruction. We found implementing this part quite challenging. Firstly, we implemented an ADT stack with functions like push and pop. We had a stack pointer (*next) which always point to the next element in the stack. A stack can be either incrementing or decrementing. Secondly, we extend the emulator and assembler to support Block Data Transfer Instructions. This instruction allows transferring multiple items in and out of the ARM processor at a time. We found it challenging to also implement the software interrupt as the understanding of how it works was difficult. 



\pagebreak

\section{Testing} \hfill \\
    Unfortunately we had a slight issue with running the testsuite at first. We didnt know that we need to pass all the test suite for emulator sections except gpio until we finished our assembler sections. We stack in debugging both part for a week and we are left behind in moving on to do part 3 and extension.
    
    We found testing our assembly code extremely difficult. There is no mechanism for unit testing, so we compromised a little and tested each function only once after it was implemented. Thus, we made sure everything works before moving on to the next function. We did this by simply branching to a test label, observing the output and branching back.
    
    This approach is very limited. We can not make sure everything still worked after making some changes. Furthermore, testing with only one LED, or changing the color of an element on screen, can not provide us with enough information. We believe that we will start implementing some test cases to test our extension we had done recently, for instance branch with link, branch and exchange and block data transfer.



\section{Group Reflection} \hfill \\
When the team was formed initially, we had trouble in starting our project due to the lack of C knowledge. This led to the issue when we felt that we had left behind the other groups most of the time. We attempted to study the lecture notes ourselves in advance and helped each other along the learning process. We failed to meet our first internal deadline although we had put much effort on working on the project. We found out that it is hard to understand the codes written by others. This caused the debugging process is taken longer than usual. 

As discussed, we believe that we need to get familiar with coding in C by further studying and referring to other resources. To make our code more readable, we think that we could put in more appropriate comments when we wrote our codes. Since we know each others well, we are confident in maintaining a good communication and interaction when we are working on our project. We used Facebook Messenger to form a group chat, where we would share what we were working and also what someone may require of another team member. Most other times, we would be sitting together in the lab, so we could converse face to face. This enabled our ideas to be transferred between each other accurately and fairly swiftly.

To increase our efficiency to finish the project, we ensure that clear group goals are set with good time management. From the real start of the project we've kept good record of group tasks, and have used git as a tool for creating issues, organizing them by milestones and assigning them to group members. This division of tasks really helped focus the group, allowing each of us to tick off tasks, giving both satisfaction and a natural organization to the project as a whole. It also prevented us from dithering in respect to the key aims of the project, something that could have been a big risk given the size of the task.

In that light we feel quite positive that as a group we can pull together on the extension, given the time from now through to the presentation. If not, we all feel like we've learned a great deal from the experience of working in a team, almost more so due to the difficulties we've experienced and overcome.

\pagebreak

\section{Individual Reflections}

\begin{quote}
\subsection*{Ben Sheng Tan}
``\textit{Overall, I consider this group project a great learning experience.  I am quite satisfied with the group management and communication. Everyone made an effort to finish our project successfully, which was really important. As four of us have no prior experience in C coding, the experience of learning new knowledge and solving the problems together are entertaining and reflective.  When someone was losing interest, there was always somebody else ready to motivate the other. From both of the WebPA feedback, I feel glad with the positive comments left by other teammates. } \\\\

\textit{This project taught me the importance of splitting the work, assisting each other, and discussing every problem with the group. As a leader, I did my best to show my enthusiasm and interest in the project to motivate my team-mates. There were good times, and times when we were debugging, but in the end we achieved our plans, so I am satisfied with the group in which I was working.}"

\end{quote}

\begin{quote}
\subsection*{Adanna Akwataghibe}
``\textit{ I think I fitted into the group very well. All the group members are friends and most of us have worked together before, so we are used to each others' working style. Whilst working in the group, it is important to maintain good communication and in my group, I always knew what each member was doing which means that we were very good at communicating with each other.}

\textit{I enjoyed working in this group because if I struggled at any part of the project, there was always someone to ask in the group who was willing to help – and this also goes for everyone. I think my strength in the team is the ability to listen to criticism and when given an instruction, I do it to the best of my ability. I would say that my weakness is always trying to solve a problem without help. I have a tendency to want to stay on a problem until it gets to the point where there nothing I can do. So, working in this group has taught me the importance of asking for help and working as a team as this is actually more efficient and that way everyone learns about the particular problem and now know how to solve it. }

\textit{Working in this group has taught me the importance of asking for help and working as a team as this is actually more efficient and that way everyone learns about the particular problem and now know how to solve it.}"
\end{quote}


\begin{quote}
\subsection*{Lan Yi}
``\textit{Working in a group has been an exciting and challenging experience for me. It is quite different from working on my own and involves not only hard work but also requires sufficient communication and co-operation skills. It has been proven that communication is my most notable weak point when work as part of a group.}

\textit{To take a panoramic view of our project, the fact is obvious that the frequent communication between group members benefits the work process a lot. We can cover the shortage or mistakes made by others immediately. Especially as for me, having the task on the part that often need the combination with other functions, it is undoubtedly necessary to exchange information with others. I feel that I understood and completed my own part itself successfully, however, due to the lack of understanding on other parts at the beginning, the program sometimes gets in trouble when we merge. At the same time, as the ability of we four people is limited, when problems shows up more discussion with other people, like other course mates and helpers, is still need.}"
\end{quote}

\begin{quote}
\subsection*{Ching Yin Wong}
``\textit{I enjoyed working in a group. The communication skill is strong among members. Having a group chat on social media like Facebook allow us to contact each other quickly when needed. Moreover, we have group meeting about 6 times a week so as to make sure that everyone can get involved in this project. Group meetings are always held in a friendly atmosphere. Members respect each other’s ideas and suggestions, which is one of the most important aspects when working in a group.}

\textit{I am glad that every members are willing and equally involved in the project. Although most of us did not know anything about C before the project, we tried our best to complete it and we enjoyed it. Due to the fact of lacking programming skills in C at the start of the project, we cherish all the lab session time to ask questions. }

\textit{Gaining from past experiences, I have learnt to voice out more with my opinions which I believed I have developed this aspect better in the ARM project. I think this is the main weakness I should keep improving for future group work.}

\textit{The main thing that I have learnt from the project and will retain for future ones is to get engage in the project as much as possible. Getting all members to enjoy the project result in better team spirit and a higher chance of producing innovative ideas.}"
\end{quote}


\end{document}
