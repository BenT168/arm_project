\documentclass[11pt]{article}

\usepackage{fullpage}

\begin{document}

\title{ARM Checkpoint Report }
\author{Group 9 : Ben Sheng Tan, Adanna Akwataghibe, Lan Yi and Ching Yin Wong}

\maketitle

\section{Work Splitting between team members}


The first meeting was hold after the introduction lecture of the project last
Thursday. The discussion was started with the analysis of the project
specification. We started splitting the work when every team member had some
basic understanding on how the project works. Before working on the project,
each of us had done some reading on the C programming.  We decided to have a
header file for all the bit manipulation functions planned to focus
on the implementation of first part before moving on to the second part.\\
\\
The tasks for Part 1 was being split as below:\\

\medskip

\begin{tabular}{ |p{5cm}|p{10cm}| }
\hline
Ben Sheng Tan
&  - structure and function for data processing instruction \\
& - function for decoding the instruction from pipeline \\
& - header and part of the implementation files for bitwise \\
& - header and implementation files for register \\
& - modification of MakeFile \\
\hline
Adanna Akwataghibe
&  - the emulator binary file loader and the main method \\
& - the multiply instruction in emulator and its helper methods\\
& - part of the structures of ARM\\
\hline
Lan Yi
&  - structure and function for single data transfer instruction\\
& - function for printing bits in implementation file for bitwise\\
& - part of the structures of ARM\\
\hline
Ching Yin Wong
& - structure and function for branch instruction\\
& - function for getting bits in implementation file for bitwise\\
& - function for checking condition function in emulator.c\\
& - emulator function in emulator.c\\
\hline
\end{tabular}

\bigskip

\\
The tasks of debugging and refining code were done by Ben Sheng Tan and Ching
Yin Wong while the implementation of the binary file writer and the abstract
data structure for assembler were carried out by Adanna and Lan Yi at the same
stage.\\

\pagebreak

\section{Structure for Emulator}

For part 1 , we have broken into several sections in emulate.c and several
files in library, as listed below:\\

\begin{tabular}{ |p{5cm}|p{10cm}| }
\hline
\multicolumn{2}{|c|}{src/emulate.c} \\
\hline
Macro & contains functions for 8 bits and 32 bits in reading and writing
from/to memory and the inclusive header files for register, which contains
the info about REGISTER READ/WRITE . \\
\hline
Function Prototype &  contains code lines with the function name, parameters
and the result it returns. \\
\hline
Binary File Loader & contains the function to read binary file reader or
otherwise, read the arm program \\
\hline
Core & contains, mainly, the pipeline simulation function - a series of
functions for easier flag operations - an instruction decoder which
identifies the instruction type based on the specified bits - a function that
checks the 4-bit conditions at the beginning - a function to print the arm
state \\
\hline
Main & contains main function for running the file \\
\hline
Instructions  & functions for initialising the constant integer, shifted registers
and checking if operand is an expression or a register and shift register. \\
\hline
Main & contains main function for running the file\\
\hline
\end{tabular}

\bigskip
\\

\begin{tabular}{ |p{5cm}|p{10cm}|  }
\hline
\multicolumn{2}{|c|}{ src/library  } \\
\hline
arm11.h & Pipeline, state of arm, register definition and flag definition \\
\hline
instruction.h & Structure of the instructions, type of shift and definition of
mnemonic\\
 \hline
bitwise.h & declare the functions written in implementation file
and define some essential methods for bit \\
\hline
bitwise.c & functions for printing bits, getting bits, rotating left and right\\
\hline
register.h  &  declare the functions written in implementation file, define the
register write and register read and the structure of CPSR flag \\
\hline
register.c & shifting functions for immediate register and shift register \\
\hline
\end{tabular}

\pagebreak

\section{Group Dynamics}

The group has worked very well so far with strong communication with each other
. Although we had a few bumps on our different programming styles, we had
refined and restructured the codes when we merged our works for part 1. As most
 of the group members are unfamiliar with git, we spent a day on git to get
used to the environment working using git. \\\\
Below are some of the problems we faced and solutions we found when carrying
out our project:\\

\begin{tabular}{ |p{5cm}|p{10cm}| }
\hline
Problems
& - using pointers correctly \\
& - allocating memory correctly (when using malloc and calloc)\\
& - dealing with merge conflicts \\
& - dealing with segmentation fault \\
\hline
Solutions
& - practice coding using pointers\\
& - the dynamic memory allocation and remembering to free memory afterwards\\
& - telling team members when pushing of pulling from git \\
& - using valgrind to check for leaks is extremely useful.\\
\hline
\end{tabular}

\section{Future Task Collaboration}

For Part two (assembler), task allocation was done among all the team
members. As all the group members are unfamiliar with coding in C, we will
practice more to get used to the concept of pointers and structure. The real
challenge will be the multiples coding styles in C programming, therefore, we
will ensure that the similar style is applied to prevent any confusion.\\

Below are some of the precautions we will beware of:\\

\begin{tabular}{ |p{5cm}|p{10cm}| }
\hline
Precautions
& - merge conflict can be a pain and could take a long time to
sort out, so communication with team is  very important to prevent this\\
& - allocate memory correctly to avoid segmentation fault\\
\hline
 \end{tabular}

\end{document}
